%\documentclass[12pt]{scrreprt}
\documentclass[12pt]{report} 

% language may be romanian or english (default is english)
% type may be bachelor or master (default is bachelor)
\usepackage[language=romanian, type=bachelor]{style}

%\geometry{a4paper,top=2.5cm,left=3cm,right=2.5cm,bottom=2.5cm}
%in style
%controlling the appearance of your headers and footers
\usepackage{fancyhdr}
\usepackage{graphicx}
\usepackage{float}
\usepackage{svg}
\usepackage{combelow}
\usepackage[utf8]{inputenc}
\usepackage[T1]{fontenc}
\pagestyle{fancy}
\lhead{}
\chead{}
\renewcommand{\headrulewidth}{0.2pt}
\renewcommand{\footrulewidth}{0.2pt}

% Attributes for preventing the overlapping of the text written with monospace font (\texttt)
\tolerance=1
\emergencystretch=\maxdimen
\hyphenpenalty=10000
\hbadness=10000

\begin{document}

\specialization{INFORMATICĂ}	
\title{Distribuția streamurilor audio-video în sesiuni WebRTC}					   
\author{Haja Florin-Gabriel}											
\supervisor{Lect. dr. Sterca Adrian-Ioan}				
				
\maketitle


\newpage
\thispagestyle{empty}
\mbox{}


\newpage
\pagenumbering{roman} 

\cleardoublepage
ABSTRACT
\vspace{0.5cm}	
\hrule
\vspace{0.5cm}	
%\cleardoublepage

%\par Această lucrare este un studiu în profunzime al protocoalelor de comunicare folosite de proiectul WebRTC și al modului în care conținutul audio-video poate fi distribuit în cadrul a mai multor clienți. Odată ce oamenii folosesc Internetul în scop productiv din ce în ce mai mult, este crucială asigurarea faptului că informația este transmisă pe cât de fluent posibil. Deoarece unele conexiuni la Internet au o lățime de bandă limitată, este mai dificilă gestionarea sesiunilor de apel video cu mulți utilizatori.
%\par This thesis is an in-depth study of communication protocols used by
%\par Putem opta ori pentru un server media care să gestioneze livrarea datelor pentru noi, ori pentru a conecta fiecare client cu toți ceilalți, într-o manieră a topologiei mesh (engl. plasă). De exemplu, în cadrul participării la un curs, se poate observa un scenariu de broadcasting, în care profesorul furnizează conținutul, iar studenții doar îl observă, așadar datele sunt transmise într-o singură direcție, iar topologia mesh nu are un beneficiu real. În locul servirii tuturor studenților deodată, putem servi câțiva, aceștia trimițând repetat streamul mai departe, până când toți pot să vizioneze ceea ce predă profesorul. Se poate observa o topologie arborescentă. Nu mai este nevoie de un server media, însemnând costuri de mentenanță mai mici, cu un compromis din partea clienților al lățimii de bandă folosite.

\par This thesis is an in-depth study of the communication protocols used by the WebRTC project and the way the audio-video content can be distributed across multiple clients. As people are using the Internet productively more and more, it is crucial to ensure the information is transmitted as fluidly as possible. Since some Internet connection may be limited in bandwidth, it is more difficult to handle video call sessions with many users. 
\par We can opt for using a media server that handles the data delivery for us, or to connect each client with the other, in a mesh topology fashion. When attending a class, for example, we are noticing a broadcasting scenario, as the teacher is providing the content and the students are only looking at it, thus the data can be streamed in one direction, and the mesh topology having no real purpose. Instead of serving all the students at once, we can serve a few, which will forward the stream to others, repeatedly streaming until everyone is able to watch what is the teacher showing. We notice a tree topology. A media server is no longer required, meaning cheaper maintenance costs, but sacrificing the low bandwidth each client is using.


\tableofcontents


\newpage
\pagenumbering{arabic}

%\addcontentsline{toc}{chapter}{Introducere}
%\addcontentsline{toc}{chapter}{Introduction}

\chapter{Introducere}
\label{intro}
\indent \par În zilele noastre, comunicarea este îndeosebi importantă. Iar pentru că distanța între persoane poate să fie mare, trebuie apelat la soluțiile de telecomunicații. Din fericire, tehnologiile au evoluat continuu și au devenit din ce în ce mai accesibile, în așa fel încât oricine care dispune de un telefon sau de un laptop, dar și de o conexiune la Internet, poate să ia legătura cu oricine, să vorbească, să se vadă și să colaboreze la diversele idei pe care le au. 
\indent \par De asemenea, impunerea unei multitudini de standarde a dus la facilitarea creării de noi platforme de comunicare, în așa fel încât nici nu mai este nevoie de a instala local vreo aplicație.
\indent \par Însă datorită creșterii populației și a cererii crescânde în acest domeniu, într-o gamă largă de contexte (personale, profesionale, academice, preuniversitare, chiar și guvernamentale), sunt inevitabile discrepanțele între calitatea diverselor conexiuni la Internet și a performanțelor pe care fiecare dispozitiv le are, așadar resursele trebuie distribuite cât mai eficient, cu scopul de a evita compromisurile în privința calității imaginii și a sunetului.
\indent \par În această lucrare, voi prezenta o posibilă optimizare într-un context des întâlnit în mediul școlar, în care o singură persoană transmite conținut audio-video, pe care îl distribuie unui grup (mai restrâns sau mai larg), în așa fel încât să nu fie dezavantajați cei care dispun de conexiuni mai slabe la Internet. Voi pune, de asemenea, la dispoziție o aplicație prin care voi demonstra această optimizare, folosind standarde din industrie precum WebRTC și WebSocket, implementată cu framework-uri moderne precum React și Socket.IO. Deoarece această optimizare presupune o topologie proprie, voi pune la dispoziție și serverul prin care va fi posibilă legătura între participanți, implementat cu Node.js, o tehnologie modernă și accesibilă.
\indent \par În primul capitol, acesta, este prezentată tema lucrării și posibilul său potențial.
\indent \par Al doilea capitol prezintă istoria videoconferințelor, care conturează momente importante din evoluția videotelefoniei, precum și tranziția către digital și standarde în industrie.
\indent \par Al treilea capitol descrie detaliat protocoalele prin care proiectul WebRTC a fost făcut posibil, precum și interfața prin care dezvoltatorii pot construi aplicații folosind protocoalele puse la dispoziție.
\indent \par Capitolul patru ilustrează modalitatea prin care aplicații populare precum Microsoft Teams și Zoom implementează aceste tehnologii.
\indent \par În capitolul cinci este inclus un studiu de caz, prin care se analizează în profunzime topologiile pe care aplicațiile moderne le folosesc pentru o comunicare eficientă, avantajele și dezavantajele lor.
\indent \par Capitolul șase conține o descriere a aplicației puse la dispoziție de mine, cazuri de utilizare, precum și topologia pe care o folosesc pentru optimizare.
\indent \par Capitolul șapte încheie lucrarea cu observații asupra posibilelor îmbunătățiri ale aplicației personale, precum și cu concluzii asupra distribuției streamurilor audio-video în sesiuni WebRTC.
\chapter{Istoria conferințelor video}
\label{sec:ch2}

\section{Era analogică}
\label{sec:ch2sec1}

\indent \par Primele concepte ale transmiterii de imagini datează încă din secolul XIX, contemporane cu brevetarea telefonului de către Alexander Graham Bell în data de 14 februarie 1876 și prezentarea acestuia la Expoziția Centenară de la Philadelphia în același an. Un articol publicat în 30 martie 1877 de către The New York Sun menționează termenul \textit{electroscop} în acest context, prin care prezintă posibilele aplicații ale unui dispozitiv prin care se pot observa imagini din orice colț al lumii. De asemenea, scriitori precum Abbé Moigno și Louis Figuier, prezintă un concept foarte similar, creând termenul \textit{telectroscop}, atribuindu-i-l eronat lui Graham Bell \cite{Electroscope1878, Telectroscope1877}. 
\begin{figure}[b]
    \centering
    \includegraphics[width=6.99cm]{figures/picturephone.jpg}
    \caption{Primul model de Picturephone, care a fost instalat în cabine în 3 orașe din Statele Unite}
\end{figure}
\indent \par Însă primul prototip funcțional a fost demonstrat public în 7 aprilie 1927 într-o aulă din New York. Herbert Hoover, pe atunci secretarul comerțului al Statelor Unite, a apărut pe ecran din Washington D.C., fiind observat de către oficialii companiei AT\&T \cite{Videophone1927}. Apelul era unidirecțional pentru imagine și bidirecțional pentru sunet, datorită liniilor telefonice deja existente. Mai târziu, la Expoziția Mondială de la New York din 1964, AT\&T a prezentat primul Picturephone, la care publicul putea să efectueze apeluri bidirecționale către o cabină telefonică similară montată la Disneyland, folosind conexiune VHF sau UHF prin cablu \cite{Avoira2020}. Acesta a fost indisponibil pentru uz casnic până în 1970, când aceeași companie a lansat un Picturephone îmbunătațit, disponibil printr-un serviciu lunar. Din cauza costului ridicat de 160 de dolari pe lună (echivalent a 1000 de dolari în 2020), nu a prezentat interes, serviciul fiind desființat în 1973 \cite{Videophone1927}.

\section{Era digitală și tranziția către Internet}
\label{sec:ch2sec2}

\indent \par În anii '80, diverse companii, printre care Mitsubishi, au comercializat diverse videotelefoane care puteau transmite imagini statice prin intermediul rețelei telefonice deja existente, fapt care permitea folosirea modemurilor existente cu rate de transfer între 2,4 și 9,6 kilobiți pe secundă. Evoluția algoritmilor digitali de compresie a imaginilor și a lățimii de bandă, a permis ca AT\&T să mai încerce o dată cu VideoPhone 2500, care, de asemenea, nu a avut succes, costând inițial 1500 de dolari în 1992 \cite{Borth98}.
\indent \par Apariția protocoalelor Internet și a tehnicilor și mai avansate de compresie audio-video a permis ca imaginea și sunetul să fie transmise în pachete mult mai mici, rezultând în costuri mult mai reduse. Simultan cu dezvoltarea rețelei ARPANET (precursoarea Internetului), destinată inițial pentru transfer de date, s-a discutat problema comunicării în timp real prin voce \cite{RFC741}. Așadar, s-a propus în decembrie 1973 implementarea Network Voice Protocol (NVP), care a venit cu următoarele considerente: separarea semnalelor de control de traficul de date, evitarea retransmiterii pachetelor pierdute, adaptabilitate la condițiile variabile ale rețelei, precum și independența gestionării resurselor de către fiecare sistem, dar și de către protocoalele de nivel mai jos\cite{RFC741}.
\indent \par PictureTel a jucat un rol important în evoluția comunicațiilor video. Cei doi fondatori ai săi, Brian L. Hinman și Jeffrey G. Bernstein, absolvenți ai MIT și buni prieteni, au fondat PicTel în 1984 (a fost ulterior redenumită pentru a evita confuzia cu termenul \textit{pixel}) cu sprijin financiar din partea lui Robert Sterling. În 1986, după ce compania a devenit publică, a dezvoltat algoritmul Motion Compensated Transform (MCT), care a permis reducerea unei transmisii audio-video de calitate de la 768 Kb/s la 224. Pe baza acestui algoritm, au lansat codecul C-2000 în luna iulie al aceluiași an, cu ajutorul căruia PictureTel a devenit lider în domeniul său. În 1988, un nou algoritm de compresie video a fost lansat, Hierarchical Vector Quantizing (HVQ), folosit în codecul C-3000, precum și în sistemul V-2100 \cite{Root2000}. Cu ajutorul acestuia, lățimea de bandă folosită a fost redusă la jumătate, comparativ cu algoritmul MCT. În anii '90, a colaborat la crearea standardului H.320 pentru videoconferințe prin rețeaua ISDN, precum și a standardului H.323 pentru comunicare audio-video prin Internet pe baza protocolullui TCP, cel din urmă folosit pentru a crea produsul software LiveLan \cite{Root2000}.
\indent \par O dată cu migrarea videoconferințelor către Internet, au început să apară soluții gratuite, potrivite și consumatorilor de rând, care să nu ceară echipament auxiliar, pe lângă un computer și o cameră web. Una din primele aplicații influente de acest fel este realizată de Tim Dorcey la Universitatea Cornell și se numește CU-SeeMe, apărut prima dată în 1992 pentru sistemele Macintosh \ref{CUSeeMeMac}. Apelurile cu doi participanți pot fi inițiate de către unul din ei conectându-se direct la celălalt, în timp ce apelurile cu mai mulți participanți necesită conectarea fiecăruia la un \textit{reflector} CU-SeeMe \cite{Dorcey95}. \textit{Reflectorul} este un software destinat sistemelor UNIX ce are menirea de a a redistribui fluxurile de pachete, principala sa motivație fiind lipsa abilităților de multicast pe Macintosh de la vremea respectivă \cite{Dorcey95}. Pentru transmisie, imaginea se împarte în pătrate de dimensiune 8x8, acestea fiind selectate pentru trimitere doar în cazul în care gradul de similitudine este suficient de ridicat \cite{Dorcey95}. Acest grad este calculat ca sumă a tuturor diferențelor absolute între fiecare pixel de pe aceeași poziție, la care se aplică o penalizare multiplicativă pentru diferențele între pixelii apropiați \cite{Dorcey95}. Mai târziu, a adoptat standardul H.323, care s-a regăsit și pe alte produse, precum Microsoft NetMeeting (începând cu versiunea 2) \cite{Vidconf, Perey2000}.
\begin{figure}[H]
    \centering
    \includegraphics[width=9.1cm]{figures/cu-seeme.jpg}
    \caption{Aplicația CU-SeeMe rulând pe Macintosh}
    \label{CUSeeMeMac}
\end{figure}
\indent \par În 1994, un moment important a fost lansarea primei camere web destinate consumatorilor, Connectix QuickCam \ref{QuickCam} \cite{Wolfe2019}. Putea captura imagini doar în 16 culori la 15 cadre pe secundă și a fost inițial compatibilă doar cu Macintosh \cite{Wolfe2019}. Un an mai târziu, a fost lansată și varianta pentru Windows. Însă calitatea redusă a imaginii nu a fost un impediment pentru persoanele aflate la distanță de a se putea simți aproape una de cealaltă, de a putea interacționa, de a trăi evenimente aflate în diverse colțuri ale lumii \cite{QuickCam94}. Ulterior, au apărut variante îmbunătațite de QuickCam, cu captură de imagini color la rezoluții mai mare, dar și cu conectivitate paralelă și USB.
\begin{figure}[H]
    \centering
    \includegraphics[width=5.7cm]{figures/quickcam.jpg}
    \caption{Connectix QuickCam, prima cameră web comercială}
    \label{QuickCam}
\end{figure}
\indent \par În paralel s-au dezvoltat o multitudine de standarde menite de a reduce și mai mult lățimea de bandă folosită, precum și de a ușura crearea de noi platforme de comunicare. În 1997 a fost creat Session Initiation Protocol (SIP), prin care utilizatori pot fi invitați la participarea într-o sesiune multimedia, sau chiar servere prin care se poate înregistra sesiunea \cite{rfc2543}. Este o combinație dintre Session Invitation Protocol și dintre Simple Conference Invitation Protocol, prin care se încearcă potrivirea lui SIP pe o infrastructură mai flexibilă care include și alte protocoale, precum SDP, HTTP și RTSP \cite{rfc2543}. Poate funcționa atât peste TCP, cât și peste UDP, pentru o performanță îmbunătațită \cite{rfc2543}.
\indent \par Codecurile video au continuat de-a lungul timpului să evolueze. În 2003, Unitatea Internațională a Telecomunicațiilor (ITU - Internation Telecommunication Union) a făcut public standardul H.264, destinat codării audio-video în aplicații precum televiziunea prin cablu și satelit, videoconferințe prin Internet, dar și filme stocate pe medii de stocare optice \cite{H.264}. Folosește tehnici precum partiționarea imaginilor în macroblocuri, precum și codificare predictivă bidirecțională. A fost dezvoltat împreună cu ISO MPEG (Moving Picture Experts Group) \cite{H.264}.
\indent \par Între timp, au apărut platforme de mesagerie instantă precum AOL Instant Messenger (AIM) în 1997, Yahoo! Messenger în 1998, MSN Messenger în 1999 \cite{Wolfe2019}. Toate au suportat apeluri video în 2003, folosind, în principiu, SIP pentru a iniția apelurile și RTP pentru transportul pachetelor. Apple a lansat în 2002 un client pentru AIM numit iChat, folosind implementarea oficială a protocolului furnizată de America Online \cite{Wolfe2019}. iChat a primit suport pentru codecul H.264 în 2004, oferind calitate mai bună decât predecesorul său, H.263 \cite{Royal2007}.
\indent \par Însă mult mai popular pentru apelurile video a devenit Skype, care, când a fost lansat în 2003 de KaZaa, permitea apeluri gratuite în grup de 25 de persoane. Promitea, la momentul respectiv, calitate a vocii superioară față de alternativele Yahoo și MSN, dar și o traversare facilă a NAT-urilor și firewall-urilor \cite{Baset2004}. Singurul rol al serverului central era doar de login, informațiile utilizatorilor fiind păstrate într-o manieră descentralizată, prin intermediul nodurilor și supernodurilor \ref{SkypeNet} \cite{Baset2004}. Un nod ce dispune de suficientă putere de procesare și lățime de bandă fi candidat pentru a fi supernod \cite{Baset2004}. Se presupunea că fiecare nod folosește o variație a protocolului STUN pentru a comunica pe rețeaua suprapusă Skype \cite{Baset2004}. De asemenea, fiecare nod își păstra o listă de supernoduri cu care să facă legătura - această listă este denumită \textit{host cache} și este păstrată în registrul sistemului de operare Windows \cite{Baset2004}. Ulterior, Skype a fost cumpărat de Microsoft în 2011 și a renunțat la topologia cu supernoduri pentru o scalabilitate mai bună, folosind, în schimb, serverele Microsoft, cu toate că au existat suspiciuni privind securitatea serviciului \cite{Whittaker2013}.
\indent \par Apoi, a urmat înglobarea unei colecții de protocoale precum SDP, STUN, TURN și RTP, precum și a unor codecuri precum VP8, G.711 și iLBC într-un proiect numit WebRTC, care vine inclus cu majoritatea browserelor moderne și pe baza căruia s-au construit majoritatea aplicațiilor moderne precum Microsoft Teams, Zoom, Jitsi Meet și OBS.Ninja, care s-au dovedit esențiale în mediul profesional și academic pe parcursul epidemiei de coronavirus începută în 2019 și care continuă și în 2021.
\begin{figure}
    \centering
    \scalebox{0.65}{\input{figures/skypenetwork.pdf_tex}}
    \caption{O reprezentare a topologiei de rețea folosită de Skype}
    \label{SkypeNet}
\end{figure}
\chapter{WebRTC}
\label{chap:ch3}

\indent \par Acest capitol are ca scop introducerea în principiile de bază și în componentele proiectului WebRTC, cu menirea de a putea înțelege conceptele următoare. Acoperă API-urile comune, precum și principiile de rețelistică folosite pentru comunicare calitativă, cu latență mică.

%\section{Istoria}
%\label{sec:ch2sec1}
% TODO mută în primul paragraf
\indent \par WebRTC a fost dezvoltat ca un standard de către World Wide Web Consortium (W3C) și de către Internet Engineering Task Force (IETF), ca o colecție open-source de standarde și protocoale. Tehnologiile necesare, precum codecuri și algoritmi de anulare a ecoului, au fost dezvoltate de către o companie suedeză numită Global IP Solutions (GIPS), care a fost mai târziu cumpărată de Google în mai 2010 \cite{WebNSM2017}.

\section{Arhitectura}
\label{sec:ch3sec1}

% TODO modifică partea asta
\indent \par Nu există un protocol standard definit pentru semalizare (engl. \textit{signaling}). Orice mecanism RPC care funcționează pe aplicații web poate fi aplicat, indiferent dacă este un API Web bazat pe HTTP (ex. REST sau GraphQL) sau pe WebSockets, care folosește HTTP pentru inițializare. 
\indent \par In the classic VoIP world, the dominant signaling protocol was SIP (Session Initiation Protocol), which brought a variety of features meant for 


\section{ICE și NAT traversal}
\label{sec:ch3sec2}

\section{Streaming. Protocoale RTP și SDP}
\label{sec:ch3sec3}

\section{Implementare în browser}
\label{sec:ch3sec4}
\indent \par WebRTC este furnizat cu fiecare browser web popular, precum și cu framework-urile derivate (ex. Electron). Este metoda principală de a dezvolta aplicații noi de conferințe video. Expune mai multe API-uri JavaScript ușor de folosit, fiecare responsabil de setul său specific de funcții, precum: cameră și microfon, conținutul ecranului și conexiuni de la egal la egal.

\subsection{Capturarea conținutului media}
\label{sec:ch2sec4subsec1}
\indent \par Pentru a face folosi streaming-ul audio-video cu un scop, o sursă de conținut este necesară. Media Devices este un nume comun pentru camerele și microfoanele conectate. În JavaScript, aceste device-uri sunt accesibile prin intermedul interfeței \texttt{navigator.mediaDevices}, prin care toate dispozitivele conectate pot fi enumerate, urmărite pentru schimbări, precum și deschise pentru a obține o instanță de \texttt{MediaStream} \cite{WebMedia2014}.
% Please review the last phrase, as it's taken word by word from the official WebRTC page
\indent \par Acest API este folosit prin apelul funcției \texttt{mediaDevices.getUserMedia()}, care returnează un promise al cărei funcție de resolve conține un parametru pentru \texttt{MediaStream}-ul asociat dispozitivului. Aceasta cere furnizarea ca parametru un obiect de tip \texttt{MediaStreamConstraints}, prin care constrângeri asupra sursei audio și video se pot specifica, precum și, opțional, identitatea peer, singura care are are acces la stream, unde conținutul este protejat ca și cum regulile CORS cross-origin ar fi în aplicare.
\indent \par O altă opțiune este de a partaja conținutul ecranului. Aceasta este posibilă prin intermediul funcției \texttt{mediaDevices.getDisplayMedia()}, foarte similară cu \texttt{getUserMedia}, care de asemenea returnează un promise ce rezolvă un \texttt{MediaStream}. Totuși, constrângerile video sunt diferite: alegerea între multiple opțiuni de a afișa cursorul (mereu, doar când este în mișcare, sau niciodată) și de a afișa zona capturată (întregul ecran, doar o fereastră, sau o filă din browser).
\indent \par Înregistrarea conținutului unui \texttt{MediaStream} este posibilă prin API-ul \texttt{MediaRecorder}, dar în această teză, ne vom concentra doar pe streaming.
\indent \par Vizionarea streamului o chestiune de a crea un elementul HTML \texttt{video}, setarea obiectului sursă ca fiind \texttt{MediaStream}-ul și de a furniza o funcție care să redea streamul când metadatele sunt încărcate, ca event handler în proprietatea \texttt{onloadedmetadata} din elementul \texttt{video}.

\subsection{Inițializarea unei conexiuni peer}
\label{sec:ch3sec4subsec2}

\indent \par Conexiunile peer sunt o parte a specificațiilor WebRTC care ajută la stabilirea unei conexiuni între două aplicații pe două calculatoare diferite pentru a comunica printr-un protocol peer-to-peer. Acestea pot transmite audio, video, sau date bine (cât timp clienții suportă API-ul \texttt{RTCDataChannel}) \cite{WebPeer2014}.
\indent \par Fiecare conexiune peer este gestionată de un obiect \texttt{RTCPeerConnection}, care este instanțiată prin constructorul său specificat ce primește ca parametru un \texttt{RTCConfiguration}, care definește modul în care conexiunea peer este pregătită și care ar trebui să conține informații despre serverele ICE folosite.
\indent \par Pentru inițierea comunicării, este prioritară crearea unei cereri sau unui răspuns SDP, în funcție dacă este vorba despre peer-ul ce apelează, sau despre peer-ul ce răspunde. Apelantul trebuie să trimită obiectul SDP pe care l-a creat peer-ului de la distanță printr-un canal de signaling, diferit de cel prin care se vor transmite datele. Procedura numită signaling nu are o definiție standard.
\indent \par Inițierea unui \texttt{RTCPeerConnection} din apelant cere crearea unui descriptor local, obiect de tip \texttt{RTCSessionDescription} prin metoda \texttt{createOffer()} din instanța de \texttt{RTCPeerConnection}. Va fi setat ca fiind descriptor local prin \texttt{setLocalDescription()} și va fi trimis la apelat prin canalul de signaling. Apelatul, când primește descriptorul sesiunii, va apela \texttt{setRemoteDescription()} și va crea un răspuns la cererea primită cu \texttt{createAnswer()} care de asemenea va fi trimis prin canalul de signaling. Inițiatorul apelului va primi răspunsul și îl va seta ca fiind descriptorul remote.

\subsection{Adăugarea stream track-urilor}
\label{sec:ch3sec4subsec3}

\indent \par După crearea unei instanțe de \texttt{RTCPeerConnection}, track-urile de stream trebuie adăugate. Revenind la dispozitivele media și la obiectele sale \texttt{MediaStream}, track-urile sunt conținute într-o listă accesibilă din funcția \texttt{getTracks()}. Acestea pot fi adăugate la conexiunea peer cu \texttt{addTrack}, care cere instanța streamului \cite{WebStream2014}.
\indent \par Peer-ul care răspunde nu conține o instanță proprie de \texttt{MediaStream} care să conțină track-urile, așadar trebuie să creeze una și să adauge track-urile remote la ea. Un handler de evenimente numit \texttt{ontrack} trebuie adăugat. Gestionează câte un track odată. În acest caz, handler-ul va adăuga la instanța nou creată de \texttt{MediaStream}, care va fi setată ca obiectul sursă al elementului \texttt{video}.

\subsection{Obținerea candidaților ICE}
\label{sec:ch3sec4subsec4}

% TODO 
\indent \par Schimbul de informație de conectivitate este obligatoriu înainte ca doi peers pot comunica. Condițiile rețelei pot fi variabile în funcție de un număr de factori (ex. ascunderea adreselor sale IP prin NAT), așadar un intermediar este folosit pentru a descoperi candidații pentru conectarea la peer. Acesta este un serviciu extern și se numește ICE (Interactive Connectivity Estabilishment) și se bazează pe un server STUN sau pe un server TURN. Indirect, serverele STUN (Session Traversal Utilites for NAT) sunt folosite în cele mai multe aplicații WebRTC, deoarece principiul lor de funcționare este mai simplu. Acestea sunt specificație în câmpul \texttt{iceServers} al obiectului de tip \texttt{RTCConfiguration} folosit la crearea obiectului \texttt{RTCPeerConnection}.
\indent \par Fiecare instanță de \texttt{RTCPeerConnection} conține un handler de evenimente pentru noii candidați, numit \texttt{onicecandidate}. Este recomandat să se adauge unul care apelează \texttt{addIceCandidate()} și trimite noii candidați la celălalt peer, care, la rândul său, îi va memora.

\chapter{Aplicații WebRTC}
\label{chap:ch4}

\section{Microsoft Teams}
\label{chap:ch4sec1}

\section{Jitsi Meet}
\label{chap:ch4sec2}

\section{Zoom}
\label{chap:ch4sec3}
\chapter{Aplicația practică}
\label{chap:ch6}

\section{Specificație}
\label{chap:ch6sec1}

\section{Design}
\label{chap:ch6sec2}

\section{Utilizare și testare}
\label{chap:ch6sec3}

\input{chapters/chapter6}
\chapter{Concluzii}
%\chapter*{Concluzii}
\label{sec:ch7}

\par Concludent, nu?

%\addcontentsline{toc}{chapter}{Concluzii}
%\addcontentsline{toc}{chapter}{Conclusions}

\bibliography{references}

\end{document}
