\chapter{Introducere}
\label{intro}
\indent \par În zilele noastre, comunicarea este îndeosebi importantă. Iar pentru că distanța între persoane poate să fie mare, trebuie apelat la soluțiile de telecomunicații. Din fericire, tehnologiile au evoluat continuu și au devenit din ce în ce mai accesibile, în așa fel încât oricine care dispune de un telefon sau de un laptop, dar și de o conexiune la Internet, poate să ia legătura cu oricine, să vorbească, să se vadă și să colaboreze la diversele idei pe care le au. 
\indent \par De asemenea, impunerea unei multitudini de standarde a dus la facilitarea creării de noi platforme de comunicare, în așa fel încât nici nu mai este nevoie de a instala local vreo aplicație.
\indent \par Însă datorită creșterii populației și a cererii crescânde în acest domeniu, într-o gamă largă de contexte (personale, profesionale, academice, preuniversitare, chiar și guvernamentale), sunt inevitabile discrepanțele între calitatea diverselor conexiuni la Internet și a performanțelor pe care fiecare dispozitiv le are, așadar resursele trebuie distribuite cât mai eficient, cu scopul de a evita compromisurile în privința calității imaginii și a sunetului.
\indent \par În această lucrare, voi prezenta o posibilă optimizare într-un context des întâlnit în mediul școlar, în care o singură persoană transmite conținut audio-video, pe care îl distribuie unui grup (mai restrâns sau mai larg), în așa fel încât să nu fie dezavantajați cei care dispun de conexiuni mai slabe la Internet. Voi pune, de asemenea, la dispoziție o aplicație prin care voi demonstra această optimizare, folosind standarde din industrie precum WebRTC și WebSocket, implementată cu framework-uri moderne precum React și Socket.IO. Deoarece această optimizare presupune o topologie proprie, voi pune la dispoziție și serverul prin care va fi posibilă legătura între participanți, implementat cu Node.js, o tehnologie modernă și accesibilă.
\indent \par În primul capitol, acesta, este prezentată tema lucrării și posibilul său potențial.
\indent \par Al doilea capitol prezintă istoria videoconferințelor, care conturează momente importante din evoluția videotelefoniei, precum și tranziția către digital și standarde în industrie.
\indent \par Al treilea capitol descrie detaliat protocoalele prin care proiectul WebRTC a fost făcut posibil, precum și interfața prin care dezvoltatorii pot construi aplicații folosind protocoalele puse la dispoziție.
\indent \par Capitolul patru ilustrează modalitatea prin care aplicații populare precum Microsoft Teams și Zoom implementează aceste tehnologii.
\indent \par În capitolul cinci este inclus un studiu de caz, prin care se analizează în profunzime topologiile pe care aplicațiile moderne le folosesc pentru o comunicare eficientă, avantajele și dezavantajele lor.
\indent \par Capitolul șase conține o descriere a aplicației puse la dispoziție de mine, cazuri de utilizare, precum și topologia pe care o folosesc pentru optimizare.
\indent \par Capitolul șapte încheie lucrarea cu observații asupra posibilelor îmbunătățiri ale aplicației personale, precum și cu concluzii asupra distribuției streamurilor audio-video în sesiuni WebRTC.