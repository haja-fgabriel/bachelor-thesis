\chapter{Concluzii}
%\chapter*{Concluzii}
\label{sec:ch7}

\indent \par Prin această lucrare am demonstrat importanța standardizării protocoalelor ce contribuie la distribuția eficientă a streamurilor audio-video în videoconferințe, precum și a alegerii unei topologii potrivite scenariului de utilizare. Videotelefonia are o istorie îndelungată, timp în care a reușit să devină accesibilă oricui, dovedindu-și utilitatea mai ales în zilele noastre.
\indent \par Prin analiza pachetelor ICE, SDP și RTP, prin descrierea API-ului JavaScript și prin multitudinea de topologii posibile am observat că WebRTC este o librărie maleabilă și foarte apreciată atât de marile corporații, pe cât și de dezvoltatorii independenți. Iar prin implementarea aplicației punând în aplicare o topologie înlănțuită împreună cu tehnica de a redistribui stream-ul, am realizat o optimizare utilă în cazul existenței multor participanți într-un apel, care ajută la evitarea compromisului de a reduce calitatea apelului.
\indent \par O posibilă îmbunătățire ar fi extinderea topologiei de distribuție înlănțuite într-una arborescentă, ce ar profita de lățimea de bandă largă pe care unii participanți o au la dispoziție și de lungimea mai scurtă a unui drum de la broadcaster la clienții cei mai depărtați din topologie, așadar de o latență redusă.
\indent \par Aplicația poate fi extinsă cu noi funcționalități care să ajute utilizatorii să comunice mai eficient, precum un câmp de conversații text și posibilitatea ca mai multe persoane să aibă camerele pornite, împărțind toți utilizatorii în mai multe topologii arborescente și făcând legătura între arbori folosind un server SFU, combinând avantajele unei latențe mai bune, a consumului redus de resurse, precum și a lățimii reduse de bandă utilizate, păstrând, așadar, o calitate ridicată a imaginii.