
\chapter{Getting started with WebRTC}
\label{chap:ch2}

\indent \par This chapter is meant as an introduction to the main principles and components of the WebRTC project, in order to be able to understand the further concepts. It is covering the commonly used APIs, as well as the networking principles that is using for a quality, low-latency communication.


\section{JavaScript APIs}
\label{sec:ch2sec1}
\indent \par WebRTC is provided with every major modern web browser and its derived frameworks (e.g. Electron) and is the main method for developing new video conferencing applications. It exposes multiple easy to use JavaScript APIs, each responsible for its specific set of features, such as: camera and microphone, display content, peer connections, remote streams and data channels.

\subsection{Capturing media}
\label{sec:ch2sec1subsec1}
\indent \par To make use of audio-video streaming, a source of content is needed. The cameras and the microphones connected to the computer or smartphone are commonly named Media Devices. In JavaScript, they can be accessed through the \texttt{navigator.mediaDevices} interface, from which all the connected devices can be enumerated, listened for changes, as well as opened for retreiving a Media Stream instance. \cite{WebRTC2014}

% Please review the last phrase, as it's taken word by word from the official WebRTC page
\indent \par Most commonly, the way this API is used is by calling the \texttt{mediaDevices.getUserMedia()} function, that returns a promise that will resolve to a MediaStream for the matching device. It requires providing a MediaStreamConstraints object as a parameter, specifying the constrains of the audio and the video source and, optionally, the peer identity that has sole access to the stream, where the content is protected as if CORS cross-origin rules were in effect.

\indent \par Another option is to share the content of the screen. This is possible through the \texttt{mediaDevices.getDisplayMedia()} function, very similar to \texttt{getUserMedia}, that also returns a promise resolving a MediaStream. However, its video constraints are different: choosing between multiple options for displaying the cursor (always, only when in motion, or never) and the captured region (the whole screen, a given window, or a browser tab).
\indent \par Recording the content of a MediaStream is possible through the MediaRecorder API, but in this thesis, the focus will only be on streaming.
\indent \par Previewing the stream is a matter of instantiating an HTML video element, setting its source object as being the MediaStream, and providing a function which plays the stream when the metadata is loaded (set the \texttt{onloadedmetadata} property of the video element).

\subsection{Initializing a peer connection}
\label{sec:ch2sec1subsec2}

\section{Signaling}
% TODO complete this section
\indent \par There is no standard protocol defined for signaling

\section{NAT traversal}